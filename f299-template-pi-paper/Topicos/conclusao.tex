No artigo de Phukan et al (2022) foi desenvolvido um projeto de hidroponia utilizando IoT e Machine Learning. Embora o artigo seja voltado para a produção de tomates e em solo, os autores obtiveram resultados promissores e abriram possibilidades para produção em larga escala e mais variedades de culturas hidropônicas. O artigo de Suresh et al (2024) baseou-se em uma fazenda hidropônica e um sistema de IoT. Embora tenha alguma semelhança, o autor não utilizou IA para automatizar e atingir os resultados obtidos.
Por sua vez o artigo de Saraswathy et al (2020) tem uma premissa muito parecida com o nosso, integrando IA e IoT em uma fazenda hidropônica. Os autores não indicam qual a variedade hidropônica foi cultivada, também não utilizaram um aplicativo móvel para controle e monitoramento. Em contrapartida, utilizaram-se de sensores para automatizar os controles de umidade, ph, temperatura, controle de luz e fluxo de água. Os autores finalizaram seu trabalho com a possibilidade de expandir os controles automatizados para toda a produção do local de estudo, oferecendo assim grandes possibilidades ao nosso trabalho. 
Os próximos passos para o nosso trabalho são o aprimoramento do projeto para aplicação em campo.
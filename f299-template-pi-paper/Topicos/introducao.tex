A ONU (Organização das Nações Unidas) foi fundada em 1945 com o objetivo de unir as nações em prol da paz e do desenvolvimento, com base nos princípios da justiça, dignidade humana e no bem-estar de todos \cite{unric2024}. A ONU e seus parceiros trabalham para atingir os 17 Objetivos de Desenvolvimento Sustentável (ODS) que abordam os principais desafios enfrentados no Brasil e no mundo. Os objetivos envolvem ações para combater a pobreza, proteger o clima e meio ambiente, e garantir paz e prosperidade às pessoas \cite{nacoesunidas2024}. As ODS 2, ODS 9, ODS 11 e ODS 12 falam, respectivamente, de fome zero e agricultura sustentável, indústria, inovação e infraestrutura, cidades e comunidades sustentáveis e consumo e produção responsáveis.

Atualmente o agronegócio no Mercosul é responsável por aproximadamente 10\% das exportações mundiais, sendo o principal exportador de commodities agrícolas básicas \cite{AgênciaGov2024}. O agronegócio brasileiro representa aproximadamente 22\% do PIB (Produto Interno Bruto) em 2024, sendo uma das principais forças econômicas do Brasil \cite{cepea2024}.

No Estado de São Paulo o PIB do ano de 2023 fechou em R\$ 3,2 Tri, representando 30\% do PIB nacional \cite{seade2024Pib}. Deste montante, o valor percentual proveniente do agronegócio é de apenas 2\% da composição do PIB paulista \cite{seade2024Pib}. Mesmo assim, no Vale do Ribeira o agronegócio é a principal fonte da economia, com destaque para as culturas da banana e do chá \cite{govsp2024}. Sendo referência na economia local, é necessária a busca por novas opções para o desenvolvimento da economia local e do agronegócio como um todo. Entre estas opções temos a proposta de fazendas verticais, ainda inexplorada na região.

Fazenda vertical é um conceito americano criado em 1999 e vem se aperfeiçoando no decorrer dos anos \cite{chloe2021}. Consiste basicamente em um modelo de cultivo utilizando locais fechados e um ambiente controlado, combinado com técnicas como a hidroponia. Para o Vale do Ribeira é um conceito interessante visando o aumento da produção e avanço da economia local, visto que há possibilidades de aumento de até 30 vezes da produção em um tempo 70\% menor \cite{gundim2022}.

O cultivo em fazendas verticais apresenta diversas vantagens quando comparado ao modelo tradicional como: ausência de secas, alagamentos, granizo, melhor controle de pragas, economia de água (graças ao reuso) além de não degradar o solo \cite{ingram2023}. Por sua vez, a hidroponia é uma técnica que, embora não seja tão difundida no Brasil, está em contínuo crescimento. Apresenta diversas vantagens como controle no uso de nutrientes, antecipação da colheita, padrão de quantidade e qualidade dos produtos colhidos, menor incidência de pragas e doenças e racionalização no uso da energia. Uma de suas desvantagens é a necessidade de acompanhamento contínuo do sistema produtivo \cite{luz2006}. Uma das culturas normalmente observadas nas produções hidropônicas é a alface.

A alface (Lactuca Sativa L.) é uma das hortaliças mais importantes do Brasil, sendo uma das preferidas para saladas. Entre os grupos destaca-se a Solta-Crespa que corresponde a 70\% do mercado brasileiro. Combinada com a hidroponia, temos uma redução em cerca de 10 dias para colheita da alface graças aos controles de umidade e temperatura da estufa \cite{luz2006}. Mesmo assim, para melhores resultados hoje temos à disposição tecnologias como IoT e redes neurais para otimizar a produção agrícola.

A tecnologia IoT (do inglês “Internet of things”, internet das coisas) consiste em conectar objetos inteligentes à internet. Desta forma é possível a transmissão de dados importantes de forma segura \cite{carnaz2016}. É uma tecnologia que, combinada com as redes neurais, será útil para fazendas verticais, visto que torna possível o acompanhamento constante do cultivo.

Por sua vez, o conceito de redes neurais trata-se basicamente de um sistema constituído de unidades de processamento simples. Cada unidade tem propensão natural de armazenar conhecimento experimental e torná-lo disponível para uso, se assemelhando bastante com o funcionamento do cérebro humano \cite{haykin2001}. Com isso, é possível a tomada de decisões constantes a partir dos dados obtidos por IoT, algo crucial para produção de uma fazenda vertical que visa produtividade e menos desperdício.

Na literatura podemos ver a existência de propostas combinando Fazendas Verticais e IoT. Os dados obtidos podem ser utilizados por redes neurais e, consequentemente, obtenção de diferentes resultados.

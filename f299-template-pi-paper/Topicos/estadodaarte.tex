O objeto de estudo deste projeto são fazendas verticais e o uso de tecnologia. Realizando busca por artigos neste sentido, foram identificados alguns trabalhos com a mesma temática. \citet{saraswathy2020} apresentaram um trabalho muito parecido com nosso trabalho relatando sobre integração de inteligência artificial utilizando IoT em uma fazenda hidropônica. A proposta era de monitorar parâmetros de umidade, pH, temperatura, intensidade de luz e fluxo de água por meio de sensores e fazer upload dos valores para a nuvem por meio de Node MCU. Utilizando uma rede neural recorrente (RNN) do tipo Long-Short-Term Memory (LSTM) com algoritmo de previsão visando maior precisão na automação, o trabalho apresentou resultado satisfatório, sendo que não seria mais necessário que os fazendeiros acompanhassem o cultivo na área automatizada, além da possibilidade de que os erros apresentados pelo rede neural serem utilizados para automatizar toda a produção da fazenda hidropônica foco do estudo. O trabalho se assemelha com nosso projeto pela utilização de redes neurais em um ambiente de fazenda vertical, porém não utiliza aplicação móvel e monitora mais parâmetros (umidade, pH, temperatura, intensidade de luz) do que os planejados neste estudo.

\citet{phukan2022} desenvolveu um projeto sobre hidroponia utilizando IoT e Machine Learning. Com o foco em otimizar o crescimento de tomates hidropônicos e em solo, o estudo apresentou resultados inesperados com a produção hidropônica atingindo alturas 54\% maiores quando comparada com os pés plantados em solo. O autor informou que o sistema alcançou 88\% em acuracidade na previsão das ações a serem realizadas, abrindo possibilidade para a produção em grande escala e uma ampla variedade de culturas hidropônicas. O trabalho se assemelha com nosso projeto pelo uso da hidroponia, IoT e machine learning, porém diverge ao utilizar tomates e também o cultivo no solo como objetivos.

\citet{suresh2024} realizaram um estudo sobre uma fazenda hidropônica com um sistema baseado em IoT. O trabalho, focado em reduzir desperdícios e maximizar os rendimentos, não utiliza inteligência artificial. Ao invés disso, trabalhou com intervalos de valores pré-estabelecidos, monitorando resultados e enviando SMS para os usuários em casos de falha. Os autores informam que o sistema é promissor e que com o avanço da tecnologia poderá apresentar ainda mais possibilidades. O trabalho se assemelha com nosso projeto ao utilizar IoT e hidroponia, porém não utilizou rede neural ou tecnologias móveis.
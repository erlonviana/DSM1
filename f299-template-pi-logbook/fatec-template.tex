%%%% fatec-article.tex, 2024/03/10

%% Classe de documento
\documentclass[
landscape,
  a4paper,%% Tamanho de papel: a4paper, letterpaper (^), etc.
  12pt,%% Tamanho de fonte: 10pt (^), 11pt, 12pt, etc.
  english,%% Idioma secundário (penúltimo) (>)
  brazilian,%% Idioma primário (último) (>)
]{article}

%% Pacotes utilizados
\usepackage[]{fatec-article}
\usepackage{setspace}

%% Processamento de entradas (itens) do índice remissivo (makeindex)
%\makeindex%

%% Arquivo(s) de referências
%\addbibresource{fatec-article.bib}

%% Início do documento
\begin{document}

% Seções e subseções
%\section{Título de Seção Primária}%

%\subsection{Título de Seção Secundária}%

%\subsubsection{Título de Seção Terciária}%

%\paragraph{Título de seção quaternária}%

%\subparagraph{Título de seção quinária}%

%\section*{Diário de Bordo}%
\section*{Instruções para o preenchimento}
\doublespacing
\begin{enumerate}
    \item O Diário de Bordo é usado para registrar atividades, progressos, ideias e desafios enfrentados em um projeto ou durante a rotina de trabalho. Serve como um registro cronológico e detalhado das operações diárias, facilitando a organização e o acompanhamento das tarefas.
    \doublespacing
    \item Durante o registro das atividades deve-se incluir detalhes como datas, horários, descrições de tarefas, nomes de participantes e observações relevantes.  Esta documentação contínua ajuda na avaliação do progresso de projetos ou atividades, permitindo ajustes e melhorias contínuas nos processos.
    \doublespacing
    \item Para evidenciar a realização das tarefas, você poderá utilizar a criação de anexos para adicionar anotações, fotos, prints, questionários, entre outros.
\end{enumerate}

\break

 \begin{table}[]
\centering
\begin{tabular}{|l|l|l|l|l|}
\hline
Nome da Atividade & Data de início & Data de término & Responsável pela atividade & Descrição da atividade realizada \\ \hline
Definição da equipe&5/8&10/8&Todos &Definição da equipe e possíveis temas\\ \hline
Temas para projeto&11/8&17/8&Todos&Pesquisas de temas para o projeto integrador\\ \hline
Tema do projeto&18/8&24/8&Todos&Escolha do tema do projeto integrador\\ \hline
Nomeação do projeto&1/9&7/9&Todos&Definição do nome do projeto mediante pesquisa\\ \hline
Identidade visual&8/9&14/9&Todos&Realização de pesquisas para definição de identidade visual\\ \hline
Banco de dados&15/9&21/9&Leandro&Pesquisa para elaboração do Apex\\ \hline
Identidade visual&22/9&28/9&Todos&Definição da identidade visual do PI\\ \hline
Diagrama de redes&22/9&28/9&João Victor&Projeto da rede (diagrama de rede)\\ \hline
Identidade visual&29/9&5/10&João&Elaboração da identidade visual do PI\\ \hline
Artigos&6/10&12/10&Érlon&Pesquisa de artigos\\ \hline
Site&6/10&12/10&Andrei&Elaboração do HTML (site do PI)\\ \hline
Canvas&13/10&19/10&Érlon&Elaboração do Canvas do Projeto\\ \hline
Reunião&20/10&20/10&Todos&Reunião de alinhamento\\ \hline
Artigos&20/10&26/10&Érlon&Revisão de literatura para o artigo\\ \hline
Site&20/10&26/10&Andrei&Desenvolvimento do site\\ \hline
Reunião&20/10&26/10&Leandro&Desenvolvimento do Apex\\ \hline
Banco de dados&27/10&2/11&João Victor&Desenvolvimento do diagrama de banco de dados\\ \hline
Diagrama de rede&27/10&2/11&João Victor&Desenvolvimento do diagrama de redes\\ \hline
UX/UI&27/10&2/11&Andrei&Desenvolvimento do Figma\\ \hline
MVP&27/10&2/11&Leandro&Desenvolvimento do Apex\\ \hline
Pitch&3/11&9/11&João Victor&Desenvolvimento do Pitch\\ \hline
Reunião&10/11&10/11&Todos&Reunião de alinhamento\\ \hline
Artigo&10/11&17/11&Érlon&Revisão do artigo\\ \hline
Pitch&10/11&17/11&João Victor&Revisão do pitch\\ \hline
Banner&10/11&17/11&João Victor&Revisão do banner\\ \hline
Figma&10/11&17/11&Andrei&Revisão do Figma\\ \hline
Apex&10/11&17/11&Todos&Finalização dos itens do PI\\ \hline
Entrega&18/11&18/11&João Victor&Entrega dos itens do PI\\ \hline
\end{tabular}
\end{table}



\end{document}